\documentclass[11pt, a4paper]{report}

\usepackage[T1]{fontenc}
\usepackage[utf8]{inputenc}
\usepackage[english]{babel}
\usepackage[top=3cm, bottom=3cm, left=2cm, right=2cm]{geometry}
\usepackage{graphics}
\usepackage{graphicx}
\usepackage{eurosym}
\usepackage{soul}
\usepackage{graphicx}
\usepackage{amsmath}
\usepackage{relsize}
\usepackage{titlepic}



\begin{titlepage}
\newcommand*{\defeq}{\stackrel{\mathsmaller{\mathsf{def}}}{=}}
\title{English project:\\Words and the world}
\author{Gauthier ZAMBAUX and Nicolas DUBOIS}
\date{\today}
\titlepic{\includegraphics[scale=0.5]{images/telecomnancy.png}
\includegraphics[scale=1]{images/universitelorraine.jpg}}
\end{titlepage}



\begin{document}

\maketitle


\chapter*{Introduction}
\paragraph{}This report introduces the reader to an application that we have been developping for the English Course of TELECOM Nancy. This application aims at helping its users learn about the English language and its history.

\paragraph{}The \textit{Words and the world} application was created followind the realisation that learners of English are often not very well familiar with both dialectal differences and the history of the language. We have therefore decided to help the user work on those two things in one single app in order to improve their knowledge of English-speaking countries and their understanding of different accents.

\paragraph{}As a consequence, the application is devided into two parts, one that allows the user to learn slang words and idioms from various English-speaking countries and another that deals with the spread of the British empire, and therefore of the English language.

\paragraph{}This report will first introduce a user documentation written in English and then a technical documentation in French.


\chapter*{User Documentation}
\paragraph{}The \textit{Words and the world} application was designed to help you learn more about differences between English dialects and the history of the British enmpire. It is thus devided into two distinct parts that are however related in that they both deal with the spread of the English language around the world. This software was designed to be usable on most computer operating systems that can support Java applications and word offline.

\paragraph{}After launching the application, the user is presented with the home tab where he can find a brief desciption of the application and links to the language and history tabs.

\section*{The history of the British empire}
\paragraph{}The first tab of the application is designed to teach the user about the history of the spread of the British empire. When the user click on the history tab button, they are presented with a description of what they can find in this part of the app. 

\paragraph{}In the following pages is described the creation of the major British colonies, along with their end and the creation of Commonwealth.


\section*{How English is English?}
\paragraph{}This part of the application allows the user to discover differences in words and expressions between several dialects of English.

\paragraph{}The first page the user is presented to when checking the language tab shows a map of the world with the names of the countries supported by the app. To access a certain country, the user needs to click on its name.

\vspace{0.15cm}
\centerline{\includegraphics[scale=0.5]{images/languageTab.png}}

\paragraph{}For each country, the user can discover a variety of vocabulary words and idioms with definition in standard English and an example. 

\vspace{0.2cm}
\centerline{\includegraphics[scale=0.5]{images/AustraliaQuestion.png}}
\vspace{-0.35cm}

\paragraph{}They can also test their knoledge with a multiple choice test as the definition of a word does not appear unless the user wants to.

\vspace{0.08cm}
\centerline{\includegraphics[scale=0.5]{images/AustraliaWrongAnswer.png}}

\paragraph{}For each word or idiom, the play button allows the user to listen to the expression as it is pronounced using the corresponding country's pronunciation.


\chapter*{Technical documentation}


\chapter*{Conclusion}



\newpage

\appendix
\chapter*{Annexe A: }
%\begin{center}
%\includegraphics[scale=0.5]{Images/pic.png}
%\end{center}

\newpage

\chapter*{Annexe B: }



\end{document}
